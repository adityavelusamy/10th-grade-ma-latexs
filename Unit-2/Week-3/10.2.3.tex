\documentclass{article}
\usepackage{graphicx, fullpage, amsmath, amsthm} % Required for inserting images
\usepackage[utf8]{inputenc}
\title{Unit 2 - Linear Algebra: Transformations, Eigenstuff, Diagonalization \\ Week 2 - Eigenvalues and Eigenspaces}
\author{Dr. Chapman and Dr. Rupel}
\date{October 2024}
\begin{document}
\maketitle
Consider the situation where there are 3 equal sized pools of water which intermix at steady rates. $10\%$ per hour moves from the first pool to the second. $5\%$ per hour moves from the second pool to the first. $15\%$ per hour moves from the second pool to the third while $10\%$ moves from the third back to the second. $5\%$ moves from the third to the first, but none moves from the first to the third. These all balance to keep the total amounts of water in each pool the same. We wish to determine the amount of a contaminant which is in each pool after a certain amount of time, if it is introduced into a specific pool.
\section{}
Write the linear system of differential equations which models this situation. Give it both as a system involving real variables, as well as a matrix and vector system. Give the solution. At this stage you may have a matrix inside an exponent
\section{}
Find the eigenvalues and eigenvectors for the coefficient matrix and use them to diagonalize and determine the general solution for the system, the answer should be in the form $A\vec{c}$ , where $A$ is a matrix of real valued functions and $\vec{c}$ represents a vector of coefficients.
\section{}
Suppose that there is initially 1 lb of salt dumped into the first pool. What is the function for the amount of salt in each of the three pools as a function of time and what is the largest amount of salt in the third pool.
\end{document}

\documentclass{article}
\usepackage{graphicx, fullpage, amsmath, amsthm, amsfonts} % Required for inserting images
\usepackage[utf8]{inputenc}
\newcommand{\RR}{\mathbb{R}}
\newcommand{\QQ}{\mathbb{Q}}
\newcommand{\subgroup}{\subseteq}
\newcommand{\normsubgroup}{\trianglelefteq}
\title{Latex Homework 10th Grade \\ Unit 3 - Abstract Algebra - Group Homomorphisms \\ Week 3 - Quotients and Isomorphism Theorems}
\author{Dr. Chapman and Dr. Rupel}
\date{\today}
\begin{document}
\maketitle
\section{}
Prove that if $N \normsubgroup G$ and $S \leq G$, then $S \cap N \normsubgroup S$. This is used in part 2.\\
Assume $N \normsubgroup G$ and $S \leq G$.\\
$n\in S\cap N, s\in S$\\
$sns^{-1}\in S$, cause it's closed.\\
$sns^{-1}\in N$, cause all elements of $S$ are in $G$ meaning it's normal to $N$. Therefore,\\
\[S \cap N \normsubgroup S\]
\section{}
Prove the second isomorphism theorem: If $N \normsubgroup G$, and $S \leq G$, then $S/(N \cap S) \cong SN/N$.\\
Assume $N \normsubgroup G$ and $S \leq G$.\\
$s_3,s_2\in S, s_2(N\cap S)\\ \phi(s_2(N\cap S))=s_2N$\\
$s_2(N\cap S) = s_3(N\cap S)$\\
$s_2=s_3n, n\in N \cap S$\\
$s_2N=s_3N$\\
$\phi(s_2(N\cap S)) = \phi(s_3(N\cap S))$\\
$n=s^{-1}_{3}s_2$\\
So $n\in S$ cuase is product of 2 elements of $S$, so injective. Therefore, $S/(N \cap S) \cong SN/N$.
\section{}
Prove the third isomorphism theorem: If $A \normsubgroup B \normsubgroup C$ and $A \normsubgroup C$ then $A/B \cong (A/C)/(B/C)$.\\
Assume $A \normsubgroup B \normsubgroup C$ and $A \normsubgroup C$\\
Let $\phi : A/C \rightarrow A/B$ given by $\phi(aC)=aB$\\
$(A/C) / ker(\phi) \cong Im(\phi)$\\
$ker(\phi)=\{aC|\phi(aC)=B\}=\{aC|aB=B\}=\{aC|a\in B\}=\{bC\}$\\
$\phi(aC)=aB=Im(\phi)$\\
$A/B\cong (A/C)/(B/C)$
\end{document}
